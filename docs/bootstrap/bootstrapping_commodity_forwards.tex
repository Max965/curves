\documentclass{article}
\usepackage{amsmath}
\usepackage[round]{natbib}
\usepackage{url}
\usepackage{hyperref}

\title{Bootstrapping of Commodity Forward Prices}
\author{Jake C. Fowler}
\date{January 2023}

\begin{document}
\newcommand{\+}[1]{\ensuremath{\mathbf{#1}}}

\maketitle

THIS DOCUMENT IS CURRENTLY WORK IN PROGRESS

\section{Introduction}
This document presents an algorithm for the boostrapping commodity forwards, futures and swap 
contracts. 


It is applicable for linear commodity contracts contingent on either a continuous delivery
of physical commodity, financial payoff based on the average of an index throughout the
contract delivery period. 

The rest of this document will refer to just forward prices and curves, even though the same 
methodology can be applied to futures and swap instruments as well.

Also the document refers to the \emph{delivery period} of the contracts, in place of 
\emph{delivery or fixing period} where \emph{fixing period} is the set of time periods from 
which the average price of an index is used to calculate the payoff of a swap contract.

\subsection{Bootstrapping Definition and Uses}
The process of bootstrapping is to take a set of prices of forward and derive a piecewise 
flat forward curve consistent with these prices. In the typical case, the delivery periods 
of the input forward contracts will be overlapping, so boostrapping can been seen as an
algorithm for producing a set of forward prices for contiguous delivery periods. 
In this wasy, bootstrapping should in the most general case be seen as an interpolation method.

% TODO comment that bootstrapped prices can't generally be realised

\bigskip

One example of the use of a bootstrapped curve is for it to be loaded into an ETRM
(Energy Trading Risk Management) system for the valuation of a portfolio of vanilla
trades. The bootstrapping creates an unambiguous price for each delivery period in the
lowest granularity which a commodity trades, which has practical benefits over just
using the raw forward prices of traded contracts.

A piecewise flat bootstrapped curve can also be used for the conservative valuation
of a physical asset with embedded optionality.

% TODO note that not lower bound because not realisable

Finally bootstrapping can be used to derive the inputs to an interpolation method which
is then used to calculate a smoothed higher granularity forward curve.

\section{No-Arbitrage Forward Price Condition}
This section describes the no-arbitrage relationship between the set of prices of forward contracts
for a consecutive delivery period and price of a single lower granularity contract which spans 
the whole delivery period of these. It is this relationship which forms the basis of the boostrapping
algorithm.

Denote:
\begin{itemize}
    \item $F$ as a forward price of the \emph{big} contract.
    The notation does not contain the concept of the
    time when this is observed, but it can just be assumed that the current valuation 
    date is earlier than $t_s$ the start of the delivery period.
    \item $f_i$ as the forward price for the $i$th \emph{small} contract a higher granularity 
    period than $F$ has, with $i=1\dots n$
    \item $D_i$ as the discount factor from the settlement date of contract $f_i$ to the 
    valuation date. We assume that the delivery period of $f_i$ is short enough for 
    the whole contract to be settled on a single date.
    \item $w_i$ is the weighting of the delivery period of contract $i$, relative to
    the delivery period that forward price $F$ corresponds to.
\end{itemize}

To show the no-arbitrage relationship, construct a portfolio which is long one unit of the 
\emph{big} contract and short $w_i$ units of each \emph{small} contract. This portfolio
is essentially flat, does not require and upfront cash, and so should have zero PV. If
the PV were positive then a long positive would present an arbitrage, as would a 
negative PV presents an arbigrate from taking a short position in the portfolio.
Expressing the PV as the discounted cash flow from take the portfolio to settlement:

\begin{equation}
    \sum_{i=1}^{n}(F-f_i) w_i D_i = 0
\end{equation}

Which can be rearranged to:

\begin{equation}
    F = \frac{\sum_{i=1}^{n}f_i w_i D_i}{\sum_{i=1}^{n}w_i D_i}
\end{equation}

This is the no-arbigrate relationship between the \emph{big} contract and it's associated
\emph{small} contracts, and can be seen as a weighted average.

\section{Bootstrapping Algorithm}

\subsection{Linear System}


Such cases are referred to in this rest of this paper as \emph{partially overlapping contracts}
and an examples seen in real market data when a contract for a one week delivery partially
overlaps with a month.


\subsection{Introduction of Nullspace Component to Solution}
The previous section has shown an example where the least squares solution does 
not give one we desire, due to a deviation from the prices of the input contracts
where partially overlapping contracts are present.
To rectify this, we can note that in the problematic scenario, and in the typical
case of running the bootstrapping algorithm, the rank of matrix $\+A$ will be
less than it's number of columns $n$, the number of boostrapped prices. In such
cases there will be infinite solutions, and a general form of these solutions found
using the fact that any vector in the nullspace of $\+A$ can be added to any
particular solution.

\bigskip

If the solution which is closest to $\+0$ is not the good, then we need to define
a new vector $\+x^{target}$, the solution which should be closest to will be choosen.
In the case of the problematic example where all input contracts have the same price,
then $\+x^{target}$ with all elements equal to this price seems like the obvious choice.
See the section below for a discussion on choosing $\+x^{target}$.

\bigskip

Defined $\+K$ as a matrix with columns containing the orthonormal basis of the nullspace
of $\+A$, and error vector as $\+e = \+x^{target} - \+x$. We want to find a linear combination
of the columns of $\+K$ which can be added to $\+x$ to move it closest to $\+x^{target}$.
This corresponds to find the least-squares solution of $\+c$ to the following system:

\begin{equation}
    \+K\+c = \+e
\end{equation}

The least-squares solution can be found be solving the Normal Equations:

\begin{equation}
    \+K^T \+K \+c = \+K^T \+e
\end{equation}

As $\+K$ is orthonormal $\+K^T \+K = \+I$, hence $\+c$ is calculated as:

\begin{equation}
    \+c = \+K^T \+e
\end{equation}

Or, substituting for $\+e$ and multiplying out the $\+K^T$ term:

\begin{equation}
    \+c = \+K^T \+x^{target} - \+K^T \+x
\end{equation}

$\+x$ is in the Row Space of $\+A$ (see \cite{Strang}), hence orthogonal to 
any vector in the nullspace of $\+A$, so $\+K^T \+x = \+0$ and the expression 
for $\+c$ can be simplified:

\begin{equation}
    \+c = \+K^T \+x^{target}
\end{equation}

% TODO provide footnote giving more details using SVD

% TODO: move this up?

Using this result, our nullspace adjusted boostrapped prices can be written as

\begin{equation}
    \+x^* = \+x + \+K\+c
\end{equation}

Substituting for $\+x$ and $\+c$:

\begin{equation}
    \+x^* = \+A^+ \+b + \+K \+K^T \+x^{target}
\end{equation}

% TODO: check \+K \+K^T is projection into nullspace, and comment on any meaning

\subsection{Choosing The Target Vector}
In the case where all input forward prices are the same, choosing $\+x^{target}$ is
easy, but what about the more realistic case when all prices are different? This is
somewhat arbitrary, but intuition says that each element of $\+x^{target}$ should be
a function of the prices of all input contracts which span the delivery period 
each element of $\+x^{target}$ represents the bootstrapped price of. Some experimentation
has shown that taking the price of contract with minimum length of delivery period yields
good results. In the case where there is more than one contract has the same minimum length,
which in practice would be unusual, then the price of the contract which is delivers earliest 
should be used.


\section{Future Work}
More research into choice of the target vector can be done, including an analysis of both
the intuitive and mathematical reasoning behind the logic for calculating $\+x^{target}$ 
presented in this paper, as well as alternative strategies. The results of varying  
$\+x^{target}$ should be analysed with real market data to assess usage in practice.

\bigskip

The whole approach to bootstrapping in this paper should also be compared to algorithms
which perform both boostrapping and interpolation, such as \cite{Benth}. In the case of
partially overalapping contracts, the inclusion of the nullspace component and target
vector presented in this paper is used to impose a shape on the resulting boostrapped
prices, but one could argue that using a spline is more suitable due to the smooth 
structure of a spline giving a more desirable shape. A piecewise flat curve could
then be averaged back from the result of the spline, and the use of a bootstapper
to create the input to another spline algorithm is no longer needed. Two downside of
using a combined bootstrapping and interpolation algorithm can be seen. The first
is that introduces additional complication in the case where only a piecewise flat
curve is required from the bootstrapping. The second is that a higher order polynomial
is required to ensure that the number of of unknowns (polynomial coefficients) is at
least as large as the number of constraints in the linear system solved in the spline
calculation. This is because the number of piecewise polynomials being solved for can 
vary when the input contracts are allowed to overlap. The higher the order of the
piecewise polynomials, the more tendency it has to oscillate.
The author has observed undesirable oscillations when using a 4th order polynomial,
as used in \cite{Benth}, when interpolating natural gas and power forward curves. 
In practice it is unlikely that a polynomial should change it's
slope more than once when interpolating commodity forward curve data.

% TODO mathematical interpretation of partially overlapping contracts and
% projection into nullspace of the target vector

\bibliographystyle{plainnat}
\bibliography{bootstrapping_commodity_forwards}

\end{document}