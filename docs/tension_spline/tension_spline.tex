\documentclass{article}
\usepackage{amsmath}
\usepackage[round]{natbib}
\usepackage{url}
\usepackage{hyperref}
\setcounter{MaxMatrixCols}{20}

\title{Tension Spline Algorithm for Building Forward Curves}
\author{Jake C. Fowler}
\date{February 2023}

\begin{document}
\newcommand{\+}[1]{\ensuremath{\mathbf{#1}}}

\maketitle

THIS DOCUMENT IS CURRENTLY WORK IN PROGRESS

\section{Introduction}

\section{Deriving the Algorithm}
\subsection{Functional Form}
The base of the algorithm is a spline, which by definition is made up of piecewise polynomial functions.
\begin{equation}
p(t) = 
\begin{cases}
    p_1(t)\qquad \text{for}\ \quad t \in [t_0, t_1) \\
    p_2(t)\qquad \text{for}\ \quad t \in [t_1, t_2) \\
    \;\vdots \\
    p_{n-1}(t)\quad \text{for}\ \quad t \in [t_{n-2}, t_{n-1}) \\
    p_n(t)\qquad \text{for}\ \quad t \in [t_{n-1}, t_n]
\end{cases}
\end{equation}
Where $t_0 < t_1 < \hdots < t_{n-1} < t_{n}$ are the boundary points between the polynomials which make up the spline.
In the context of building a forward curve, the variable $t$ is defined as
the time until start of delivery of a forward contract.

\bigskip

The boundary points are chosen to be start of the input forward prices. It is also
assumed that the input forward prices are not for delivery periods which overlap
with any other input. Gaps between input forward contracts are permitted, in which
case a boundary point will exit for the start of the gap.

\bigskip

\begin{multline}
p_i(t) = \frac{z_{i-1} \sinh(\tau_i (t_i - t)) + z_i \sinh(\tau_i (t - t_{i-1}))}{\tau_i^2 \sinh(\tau h_i)}  \\
    + \frac{(y_{i-1} - z_{i-1}/\tau_i^2)(t_i - t) + (y_i - z_i/\tau_i^2)(t - t_{i-1})}{h_i}
\end{multline}

Where $h_i =  t_i - t_{i-1}$. 
$z_i = p^{\prime \prime}(t_i)$ and $y_i = p(t_i)$, i.e. the (as yet unknown) value of the function at
the boundary points.

The curve fitting algorithm essentially involves solving for the parameters
$z_i$, and $y_i$ for $i=0 \hdots n$.

\bigskip
In many cases the spline described above is not sufficient to derive a forward curve which
shows strong price seasonality, especially when this seasonality cannot be directly observed
in the traded forward prices. An example of this is the day-of-week seasonality for gas and
power prices, which generally are lower at the weekend when demand is lower. As such the 
function form is as follows:

\begin{equation}
    \label{eq:foward_function}
    f(t) = (p(t) + S_{add}(t))S_{mult}(t)
\end{equation}

Where the forward price for the period starting delivery at time $t$ is given by $f(t)$, which 
consists of $p(t)$ adjusted by two arbitrary seasonal adjustment functions
$S_{add}(t)$ an additive adjustment, and $S_{mult}(t)$ a multiplicative adjustment.


\subsection{Constraints}
\subsubsection{Polynomial Boundary Point Constraints}
As usual with splines, constraints are put in place that adjascent polynomials 
have equal value, first derivative, and second derivatives at the boundary points.

\subsubsection{Polynomial Value Boundary Point Equality}
To make $p(t)$ continuous we need to constrain $p_i(t_{i-1}) = p_{i-1}(t_{i-1})$. Evaluating both
of these:

\begin{eqnarray}
    \nonumber
    p_i(t_{i-1}) = \frac{z_{i-1} \sinh(\tau_i h_i) + z_i \sinh(0)}{\tau_i^2 \sinh(\tau_i h_i)}
    + \frac{(y_{i-1} - z_{i-1}/\tau_i^2) h_i }{h_i} \\
    \nonumber
    = \frac{z_{i-1}}{\tau_i^2} + y_{i-1} - \frac{z_{i-1}}{\tau_i^2} \\
    = y_{i-1}
\end{eqnarray}

\begin{eqnarray}
    \nonumber
    p_{i-1}(t_{i-1}) = \frac{z_{i-2} \sinh(0) + z_{i-1} \sinh(\tau_{i-1} h_{i-1})}{\tau_{i-1}^2 \sinh(\tau_{i-1} h_{i-1})}
    + \frac{(y_{i-1} - z_{i-1}/\tau_{i-1}^2) h_{i-1}}{h_{i-1}} \\
    \nonumber
    = \frac{z_{i-1}}{\tau_{i-1}^2} + y_{i-1} - \frac{z_{i-1}}{\tau_{i-1}^2} \\
    \nonumber
    = y_{i-1}
\end{eqnarray}

Hence, by construction, $p(t)$ is always continuous with value $y_{i-1}$ at the boundary
between $p_i$ and $p_{i - 1}$

\subsubsection{Polynomial First Derivative Boundary Point Equality}
This is to constrain $p^\prime_i(t_{i-1}) = p^\prime_{i-1}(t_{i-1})$. First finding the 
expression for $p^\prime_i(t)$:

\begin{multline}
    p_i(t) = \frac{z_{i-1} \sinh(\tau_i (t_i - t)) + z_i \sinh(\tau_i (t - t_{i-1}))}{\tau_i^2 \sinh(\tau h_i)}  \\
        + \frac{(y_{i-1} - z_{i-1}/\tau_i^2)(t_i - t) + (y_i - z_i/\tau_i^2)(t - t_{i-1})}{h_i}
\end{multline}

\begin{multline}
    p^\prime_i(t) = \frac{ -z_{i-1} \cosh(\tau_i (t_i - t)) + z_i \cosh(\tau_i (t - t_{i-1}))}{\tau_i \sinh(\tau_i h_i)}  \\
        + \frac{y_i - y_{i-1} +  (z_{i-1} - z_i)/\tau_i^2}{h_i}
\end{multline}

For clarity, rearranging this to highlight the linearity with respect to the parameters:

\begin{multline}
    p^\prime_i(t) = z_i \biggl( \frac{ \cosh(\tau_i (t - t_{i-1}))}{\tau_i \sinh(\tau_i h_i)} - \frac{1}{h_i \tau_i^2} \biggr) 
        + z_{i-1} \biggl( \frac{1}{h_i \tau_i^2} - \frac{ \cosh(\tau_i (t_i - t))}{\tau_i \sinh(\tau_i h_i)} \biggr)\\
        + y_i \frac{1}{h_i} - y_{i - 1} \frac{1}{h_i}
\end{multline}

Evaluating this about the boundary points:

\begin{multline}
    p^\prime_i(t_{i-1}) = z_i \biggl( \frac{1}{\tau_i \sinh(\tau_i h_i)} - \frac{1}{h_i \tau_i^2} \biggr) 
        + z_{i-1} \biggl( \frac{1}{h_i \tau_i^2} - \frac{ \cosh(\tau_i h_i)}{\tau_i \sinh(\tau_i h_i)} \biggr)\\
        + y_i \frac{1}{h_i} - y_{i - 1} \frac{1}{h_i}
\end{multline}

\begin{multline}
    p^\prime_{i-1}(t_{i-1}) = z_{i-1} \biggl( \frac{ \cosh(\tau_{i-1} h_{i-1})}{\tau_{i-1} \sinh(\tau_{i-1} h_{i-1})} - \frac{1}{h_{i-1} \tau_{i-1}^2} \biggr) 
        + z_{i-2} \biggl( \frac{1}{h_{i-1} \tau_{i-1}^2} - \frac{1}{\tau_{i-1} \sinh(\tau_{i-1} h_{i-1})} \biggr)\\
        + y_{i-1} \frac{1}{h_{i-1}} - y_{i-2} \frac{1}{h_{i-1}}
\end{multline}

Setting these equal:

\begin{multline}
    0 = z_i \biggl( \frac{1}{\tau_i \sinh(\tau_i h_i)} - \frac{1}{h_i \tau_i^2} \biggr) \\
        + z_{i-1} \biggl( \frac{1}{h_i \tau_i^2} -\frac{ \cosh(\tau_i h_i)}{\tau_i \sinh(\tau_i h_i)}
        - \frac{ \cosh(\tau_{i-1} h_{i-1})}{\tau_{i-1} \sinh(\tau_{i-1} h_{i-1})} + \frac{1}{h_{i-1} \tau_{i-1}^2}\biggr)\\
        - z_{i-2} \biggl( \frac{1}{h_{i-1} \tau_{i-1}^2} - \frac{1}{\tau_{i-1} \sinh(\tau_{i-1} h_{i-1})} \biggr)\\
        + y_i \frac{1}{h_i} - y_{i - 1} \bigl( \frac{1}{h_i} + \frac{1}{h_{i-1}} \bigr)
        + y_{i-2} \frac{1}{h_{i-1}}
\end{multline}

This constraint should be held for $i = 2 \hdots n$.

\bigskip

The above three equation should hold for the boundary points \\
$t \in \{t_1, t_2, \hdots, t_{n-2}, t_{n-1}\}$.

\subsubsection{Forward Price Constraint}
The most important constraints is that the derived forward curve averages back to the 
input traded forward prices.
The market inputs to the forward curve model are traded forward prices $F_i$.
Setting this equal to the average of the derived smooth curve:

\begin{equation}
    \label{eq:traded_forward_calc}
    F_j = \frac{\sum_{t \in T_j} (p(t) + S_{add}(t))S_{mult}(t)w(t)D(t)}
    {\sum_{t \in T_j} w(t)D(t)}
\end{equation}

Where $D(t)$ is the discount factor from the settlement date of delivery period $t$.
$w(t)$ is a weighting function and $T_i$ is the set of all delivery start times
for the delivery periods at the granularity of the curve being built. The weighting function has two
meanings from a busines perspective.
\begin{itemize}
    \item The volume of commodity delivered in each period. For example, an off-peak power forward 
    contract in the UK delivers over 12 hours in on weekdays, and 24 hours on weekends, hence $w(t)$ 
    would equal double for $t$ representing weekends compared to $w(t)$ when $t$ represents a weekday delivery. 
    Clock changes can also cause the total volume delivered over a day in a fixed time zone to vary 
    due to hours lost or gained. Hence $w(t)$ can be used to account for this.
    \item For swaps which only fix on certain days (usually business days) $w(t)$ can be used to account 
    for this by returning the number of fixing days in the period starting at $t$.
    For example if deriving a a monthly curve $w(t)$ would evaluate to the number of fixing days in the 
    month starting at $t$.
\end{itemize}
Equation \ref{eq:traded_forward_calc} can be transformed into an equation linear on the parameters of the piecewise polynomial by substituting
in the polynomial representation of $p(t)$:

\begin{equation}
    \sum_i \sum_{t \in T_j \cap  [t_{i-1}, t_i) } p_i(t) S_{mult}(t)w(t)D(t) = 
    F_j \sum_{t \in T_j} w(t) D(t) - \sum_{t \in T_i} S_{add}(t) S_{mult}(t)w(t)D(t)
\end{equation}



Substituting in for $p_i(t)$:
\begin{multline}
    \sum_i \sum_{t \in T_i \cap  [t_{i-1}, t_i)} \biggl( \frac{z_{i-1} \sinh(\tau_i (t_i - t)) + z_i \sinh(\tau_i (t - t_{i-1}))}{\tau_i^2 \sinh(\tau_i h_i)}  \\
    + \frac{(y_{i-1} - z_{i-1}/\tau_i^2)(t_i - t) + (y_i - z_i/\tau_i^2)(t - t_{i-1})}{h_i} + 
    S_{add}(t) \biggr) S_{mult}(t)w(t)D(t) \\
    = F_i \sum_{t \in T_i} w(t)D(t) - \sum_{t \in T_i} S_{add}(t) S_{mult}(t)w(t)D(t)
\end{multline}


Rearranging again gives a form linear with respect to the unknown polynomial coefficients
$z_i$, $z_{-1}$, $y_i$ and $y_{i-1}$.

\begin{multline}
    \sum_i \biggl(  z_i \sum_{t \in T_j \cap [t_{i-1}, t_i)} \biggl( \frac{\sinh(\tau_i (t - t_{i-1}))}{\tau_i^2 \sinh(\tau_i h_i)} 
      - \frac{t - t_{i-1}}{\tau_i^2 h_i} \biggr) S_{mult}(t)w(t)D(t) \\
    + z_{i-1} \sum_{t \in T_j \cap [t_{i-1}, t_i)} \biggl( \frac{\sinh(\tau_i (t_i - t))}{\tau_i^2 \sinh(\tau_i h_i)} 
     - \frac{t_i - t}{\tau_i^2 h_i} \biggr) S_{mult}(t)w(t)D(t) \\
    + y_i \sum_{t \in T_j \cap [t_{i-1}, t_i)} \frac{(t - t_{i-1})}{h_i} S_{mult}(t)w(t)D(t) \\
    + y_{i-1} \sum_{t \in T_j \cap [t_{i-1}, t_i)} \frac{(t_i - t)}{h_i} S_{mult}(t)w(t)D(t) \biggr) \\
    = F_j \sum_{t \in T_i} w(t)D(t) - \sum_{t \in T_i} S_{add}(t) S_{mult}(t)w(t)D(t)
\end{multline}

This constraint should be held for $j=1 \hdots n$.

\subsubsection{Matrix Form of Constraints}
%Equations \ref{eq:continuity_constraint}, \ref{eq:1st_deriv_constraint}, \ref{eq:2nd_deriv_constraint} and
% \ref{eq:price_constraint} can be expressed as the linear system $\+{Ax = b}$ where:


Start with forward price constraint as less lags (probably)

\begin{equation}
    \alpha^j_i = 
\end{equation}

Superscript is for contract, supersci

\begin{equation*}
    \begin{bmatrix}
        1 & 0 & 0 & 0 & 0 & 0 & \hdots & 0 & 0 & 0 & 0 & 0 & 0 \\
        \alpha^1_0 & \beta^1_0 & \gamma^1_0 & \delta^1_0 & \alpha^1_1 & \beta^1_1 & \hdots & 0 & 0 & \alpha^1_n & \beta^1_n & \gamma^1_n & \delta^1_n \\
        0 & 0 & 0 & 0 & 0 & 0 & \hdots & 0 & 0 & 0 & 0 & 0 & 0 \\
        \alpha^2_0 & \beta^2_0 & \gamma^2_0 & \delta^2_0 & \alpha^2_1 & \beta^2_1 & \hdots & 0 & 0 & \alpha^2_n & \beta^2_n & \gamma^2_n & \delta^2_n \\        
        \vdots & & & & & & \ddots & & & & & & \vdots \\
        0 & 0 & 0 & 0 & 0 & 0 & \hdots & 0 & 0 & 0 & 0 & 0 & 0 \\
        \alpha^n_0 & \beta^n_0 & \gamma^n_0 & \delta^n_0 & \alpha^n_1 & \beta^n_1 & \hdots & 0 & 0 & \alpha^n_n & \beta^n_n & \gamma^n_n & \delta^n_n \\
        0 & 0 & 0 & 0 & 0 & 0 & \hdots & 0 & 0 & 0 & 0 & 0 & 0 \\
        0 & 0 & 0 & 0 & 0 & 0 & \hdots & 0 & 0 & 0 & 0 & 1 & 0 
    \end{bmatrix}
    \begin{bmatrix}
        z_0 \\
        y_0 \\
        z_1 \\
        y_1 \\
        \vdots \\
        z_{n-1} \\
        y_{n-1} \\
        z_n \\
        y_n \\
    \end{bmatrix} =
    \begin{bmatrix}
        0 \\
        f_i \\
        0 \\
        0 \\
        \vdots \\
        0 \\
        0 \\
        0 \\
        0 \\
    \end{bmatrix}
\end{equation*}




\end{document}