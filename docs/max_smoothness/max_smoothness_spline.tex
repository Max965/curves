\documentclass{article}
\usepackage{amsmath}
\usepackage[round]{natbib}
\usepackage{url}
\usepackage{hyperref}

\title{Maximum Smoothness Algorithm for Building Commodity Forward Curves}
\author{Jake C. Fowler}
\date{June 2019 (Updated March 2023)}

\begin{document}
\newcommand{\+}[1]{\ensuremath{\mathbf{#1}}}

\maketitle

\section{Introduction}
This document describes the maximum-smoothness algorithm used to increase the granularity of constructed forward curves.
The rest of this document will refer to forward prices and curves, even though the same methodology can be applied to futures and swap instruments as well.

\bigskip
The methodology is an adaption of that described in \citep{Unknown}. 
This makes reference to \cite{Benth} and \cite{Lim}, the later reference which uses maximum smoothness
in the context of building interest rate forward curves.
% TODO reference online article and Benth
\bigskip

A tension parameter is introduced in this paper, which allows the interpolation to not be
based on only maximising the smoothness, but also to penalise the length of the curve. This
can be used to reduce oscillations in the interpolated curve.

\section{Deriving the Algorithm}
\subsection{Functional Form}
The base of the algorithm is a spline, which by definition is made up of piecewise polynomial functions.
\begin{equation}
p(t) = 
\begin{cases}
    p_1(t)\qquad \text{for}\ \quad t \in [t_0, t_1) \\
    p_2(t)\qquad \text{for}\ \quad t \in [t_1, t_2) \\
    \;\vdots \\
    p_{n-1}(t)\quad \text{for}\ \quad t \in [t_{n-2}, t_{n-1}) \\
    p_n(t)\qquad \text{for}\ \quad t \in [t_{n-1}, t_n]
\end{cases}
\end{equation}
Where $t_0 < t_1 < \hdots < t_{n-1} < t_{n}$ are the boundary points between the polynomials which make up the spline.
In the context of building a forward curve, the variable $t$ is defined as
the time until start of delivery of a forward contract.

\bigskip

The boundary points are chosen to be start of the input forward prices. It is also
assumed that the input forward prices are not for delivery periods which overlap
with any other input. Gaps between input forward contracts are permitted, in which
case a boundary point will exit for the start of the gap.

\bigskip

The individual polynomial functions $p_n$ themselves are of order 4, hence take the form:
\begin{equation}
p_i(t) = a_i + b_i t + c_i t^2 + d_i t^3 + e_i t^4
\end{equation}
The curve fitting algorithm essentially involves solving for the polynomial parameters
$a_i$, $b_i$, $c_i$, $d_i$, and $e_i$ for $i=i \hdots n$.

\bigskip
In many cases the spline described above is not sufficient to derive a forward curve which
shows strong price seasonality, especially when this seasonality cannot be directly observed
in the traded forward prices. An example of this is the day-of-week seasonality for gas and
power prices, which generally are lower at the weekend when demand is lower. As such the 
function form is as follows:

\begin{equation}
    \label{eq:foward_function}
    f(t) = (p(t) + S_{add}(t))S_{mult}(t)
\end{equation}

Where the forward price for the period starting delivery at time $t$ is given by $f(t)$, which 
consists of $p(t)$ adjusted by two arbitrary seasonal adjustment functions
$S_{add}(t)$ an additive adjustment, and $S_{mult}(t)$ a multiplicative adjustment.


\subsection{Constraints}
\subsubsection{Polynomial Boundary Point Constraints}
As usual with splines, constraints are put in place that adjascent polynomials have equal value, first derivative, and second derivatives at the boundary points.
These three constraints can be respectively expressed as:
\begin{equation}
    \label{eq:continuity_constraint}
    a_i + b_i t + c_i t^2 + d_i t^3 + e_i t^4 - 
    a_{i+1} - b_{i+1} t - c_{i+1} t^2 - d_{i+1} t^3 - e_{i+1} t^4 = 0
\end{equation}

\begin{equation}
    \label{eq:1st_deriv_constraint}
    b_i + 2 c_i t + 3 d_i t^2 + 4 e_i t^3 - 
    b_{i+1} - 2 c_{i+1} t - 3 d_{i+1} t^2 - 4 e_{i+1} t^3 = 0
\end{equation}

\begin{equation}
    \label{eq:2nd_deriv_constraint}
    2 c_i + 6 d_i t + 12 e_i t^2 - 
    2 c_{i+1} - 6 d_{i+1} t - 12 e_{i+1} t^2 = 0
\end{equation}
The above three equations should hold for the boundary points \\
$t \in \{t_1, t_2, \hdots, t_{n-2}, t_{n-1}\}$.

\subsubsection{Forward Price Constraint}
The most important constraints is that the derived forward curve averages back to the 
input traded forward prices.
The market inputs to the forward curve model are traded forward prices $F_i$.
Setting this equal to the average of the derived smooth curve:

\begin{equation}
    \label{eq:traded_forward_calc}
    F_i = \frac{\sum_{t \in T_i} (p(t) + S_{add}(t))S_{mult}(t)w(t)}
    {\sum_{t \in T_i} w(t)}
\end{equation}

Where $w(t)$ is a weighting function and $T_i$ is the set of all delivery start times
for the delivery periods at the granularity of the curve being built. The weighting function has two
meanings from a busines perspective.
\begin{itemize}
    \item The discount factor, because the no-arbitrage price of a forward contract is equal 
    to the discount factor weighted average of its components. However, in a low interest rate 
    environment, the discount factor can be approximated as 1.0 for all maturities, in which case 
    the discounting component can be ignored.
    \item The volume of commodity delivered in each period. For example, an off-peak power forward 
    contract in the UK delivers over 12 hours in on weekdays, and 24 hours on weekends, hence $w(t)$ 
    would equal double for $t$ representing weekends compared to $w(t)$ when $t$ represents a weekday delivery. 
    Clock changes can also cause the total volume delivered over a day in a fixed time zone to vary 
    due to hours lost or gained. Hence $w(t)$ can be used to account for this.
    \item For swaps which only fix on certain days (usually business days) $w(t)$ can be used to account 
    for this by returning the number of fixing days in the period starting at $t$.
    For example if deriving a a monthly curve $w(t)$ would evaluate to the number of fixing days in the 
    month starting at $t$.
\end{itemize}
Equation \ref{eq:traded_forward_calc} can be transformed into an equation linear on the parameters of the piecewise polynomial by substituting
in the polynomial representation of $p(t)$:

\begin{equation}
    \sum_{t \in T_i} (p_i(t) + S_{add}(t)) S_{mult}(t)w(t) = 
    F_i \sum_{t \in T_i} w(t)
\end{equation}
Rearranging:
\begin{equation}
    \sum_{t \in T_i} (a_i + b_i t + c_i t^2 + d_i t^3 + e_i t^4  + 
    S_{add}(t)) S_{mult}(t)w(t) = F_i \sum_{t \in T_i} w(t)
\end{equation}
Rearranging again gives a form linear with respect to the unknown polynomial coefficients $a_i$, $b_i$, $c_i$ and $d_i$.
\begin{eqnarray}
    \label{eq:price_constraint}
    \nonumber
    a_i \sum_{t \in T_i} S_{mult}(t)w(t) +
    b_i \sum_{t \in T_i} S_{mult}(t)w(t) t +
    c_i \sum_{t \in T_i} S_{mult}(t)w(t) t^2 + \\
    \nonumber
    d_i \sum_{t \in T_i} S_{mult}(t)w(t) t^3 +
    e_i \sum_{t \in T_i} S_{mult}(t)w(t) t^4 = \\ 
    F_i \sum_{t \in T_i} w(t) - 
    \sum_{t \in T_i} S_{add}(t)S_{mult}(t)w(t)
\end{eqnarray}

In the case where there are gaps between the input forward contract delivery periods,
the forward price constraint is simply omitted.

\subsubsection{Matrix Form of Constraints}
Equations \ref{eq:continuity_constraint}, \ref{eq:1st_deriv_constraint}, \ref{eq:2nd_deriv_constraint} and
 \ref{eq:price_constraint} can be expressed as the linear system $\+{Ax = b}$ where:

\begin{equation*}
    \+x = \begin{bmatrix}
        \+x_1 \\
        \+x_2 \\
        \vdots \\
        \+x_{n-1} \\
        \+x_n \\
    \end{bmatrix}
\end{equation*}

\begin{equation*}
    \+b = \begin{bmatrix}
        \+b_1 \\
        \+b_2 \\
        \vdots \\
        \+b_{n-1} \\
        \+b_n \\
    \end{bmatrix}
\end{equation*}

\begin{equation*}
    \+A = \begin{bmatrix}
        \+A_1 & \+0 & \hdots & \+0 & \+0 \\
        \+0 & \+A_2 & \hdots & \+0 & \+0 \\
        \vdots & & \ddots & & \vdots \\
        \+0 & \+0 & & \+A_{n-1} & \+0 \\
        \+0 & \+0 & \hdots & \+0 & \+A_n
    \end{bmatrix}
\end{equation*}

And:
\begin{equation*}
    \+{x_i} = \begin{bmatrix}
        a_i \\
        b_i \\
        c_i \\
        d_i \\
        e_i \\
    \end{bmatrix}
\end{equation*}

\begin{equation*}
    \+{b_i} = \begin{bmatrix}
        0 \\
        0 \\
        0 \\
        F_i \sum_{t \in T_i} w(t) - 
        \sum_{t \in T_i} S_{add}(t)S_{mult}(t)w(t)
    \end{bmatrix}
\end{equation*}

\begin{equation*}
    \+A_i = \begin{bmatrix}
        1 & t_i & t_i^2 & t_i^3 & t_i^4 & -1 & -t_i & -t_i^2 & -t_i^3 & -t_i^4 \\
        0 & 1 & 2 t_i & 3 t_i^2 & 4 t_i^3 & 0 & -1 & -2 t_i & -3 t_i^2 & -4 t_i^3 \\
        0 & 0 & 2 & 6 t & 12 t^2 & 0 & 0 & -2 & -6 t & -12 t^2 \\
        f_i^1 & f_i^2 & f_i^3 & f_i^4 & f_i^5 & 0 & 0 & 0 & 0 & 0
    \end{bmatrix}
\end{equation*}
Where the components in the last row are defined as:

\begin{eqnarray}
    \nonumber
    f_i^1 = \sum_{t \in T_i} S_{mult}(t)w(t) \\
    \nonumber
    f_i^2 = \sum_{t \in T_i} S_{mult}(t)w(t) t \\
    \nonumber
    f_i^3 = \sum_{t \in T_i} S_{mult}(t)w(t) t^2 \\
    \nonumber
    f_i^4 = \sum_{t \in T_i} S_{mult}(t)w(t) t^3 \\
    \nonumber
    f_i^5 = \sum_{t \in T_i} S_{mult}(t)w(t) t^4  
\end{eqnarray}

\subsection{Smoothness Criteria}
% TODO references
Maximum smoothness is typically obtained by by finding the spline 
parameters which minimise the integral of the second derivative squared. This paper deviates from
this methodology by also including the squared first derivative in the penalty function being
minimised. Both the first and second derivative terms penalise oscillations, but in different
ways. The first derivative penalises the increased total curve length of oscillations, whereas the second
derivative term penalises changes in curve direction. The non-negative tension parameter 
$\tau$ is used to control the contribution of curve length to the penalty function being minimised.
Note that although this spline includes a tension parameter, it's behaviour with respect to this
parameter is very different to \emph{usual} tension splines which will tend towards linear splines
as the tension parameter increases. The effect of the tension parameter $\tau$ in this 
algorithm is more subtle, with the curve alway remaining smooth, but with oscillations generally
becoming smaller in amplitude, but with sharper changes in direction when $\tau$ is increased.

\bigskip

Writing the penalty function and integrating:

\begin{eqnarray}
    \label{eq:11}
    \nonumber
    min \int_{t_0}^{t_n} \bigl( p^{\prime\prime}(t)^2 + \tau p^{\prime}(t)^2 \bigr) dt 
    = \sum_{i=1}^{n}\int_{t_{i - 1}}^{t_i}p^{\prime\prime}_i(t)^2 + \tau p^{\prime}(t)^2 dt \\
    \nonumber
    =\sum_{i=1}^{n}\int_{t_{i - 1}}^{t_i} \bigl( 2 c_i + 6 d_i t + 12 e_i t^2\bigr)^2 
    + \tau \bigl( b_i + 2 c_i t + 3 d_i t^2 + 4 e_i t^3 \bigr)^2 dt \\
    \nonumber
    =\sum_{i=1}^{n}\int_{t_{i - 1}}^{t_i} \biggl( b_i^2\tau + 4 b_i c_i \tau t + c_i^2(4 + 4 \tau t^2) 
    + 6 b_i d_i \tau t^2 + c_i d_i (24 t + 12 \tau t^3) + \\
    \nonumber
    c_i e_i (48 t^2 + 16 \tau t^4) + d_i^2 (36 t^2 + 9 \tau t^4) + \\
    \nonumber
    8 b_i e_i \tau t^3 + d_i e_i (144 t^3 + 24 \tau t^5) + 
     e_i^2 (144 t^4 + 16 \tau t^6) \biggr) dt \\
    \nonumber
    =\sum_{i=1}^{n} b_i^2 \tau \Delta_i^1 + 2 b_i c_i \tau \Delta_i^2  + 4 c_i^2 (\Delta_i^1 + \frac{1}{3} \tau \Delta_i^3  ) 
    + 2 b_i d_i \tau \Delta_i^3 + c_i d_i (12 \Delta_i^2 + 3 \tau \Delta_i^4 ) \\
    \nonumber
    + c_i e_i ( 16 \Delta_i^3 + \frac{16}{5} \tau \Delta_i^5 ) + d_i^2 (12 \Delta_i^3 + 
    \frac{9}{5} \tau \Delta_i^5 ) \\
    + 2 b_i e_i \tau \Delta_i^4 + d_i e_i ( 36 \Delta_i^4 + 4 \tau \Delta_i^6 ) + 
    e_i^2 ( \frac{144}{5} \Delta_i^5 + \frac{16}{7} \tau \Delta_i^7 )
\end{eqnarray}

Where $\Delta_i^j$ is defined as the difference between $t^j$ at the polynomial boundary points, 
i.e. $\Delta_i^j = t_i^j - t_{i-1}^j$.
Recognising \ref{eq:11} as a quadratic form it can be reformulating in the following matrix form:


\begin{equation}
    \label{eq:12}
    \sum_{i=1}^{n}\+{x_i^TH_ix_i}
\end{equation}
Where:

\begin{equation*}
    \+{H_i} = \begin{bmatrix}
        0 & 0 & 0 & 0 & 0 \\
        0 & \tau \Delta_i^1 & \tau \Delta_i^2 & \tau \Delta_i^3 & \tau \Delta_i^4 \\
        0 & \tau \Delta_i^2 & 4 \Delta_i^1 + \frac{4}{3} \tau \Delta_i^3 & 6 \Delta_i^2 + \frac{3}{2} \tau \Delta_i^4 & 8 \Delta_i^3 + \frac{8}{5} \tau \Delta_i^5 \\
        0 & \tau \Delta_i^3 & 6 \Delta_i^2 + \frac{3}{2} \tau \Delta_i^4 & 12 \Delta_i^3 + \frac{9}{5} \tau \Delta_i^5 & 18 \Delta_i^4 + 2 \tau \Delta_i^6 \\
        0 & \tau \Delta_i^4 & 8 \Delta_i^3 + \frac{8}{5} \tau \Delta_i^5 & 18 \Delta_i^4 + 2 \tau \Delta_i^6 & \frac{144}{5} \Delta_i^5 + \frac{16}{7} \tau \Delta_i^7
    \end{bmatrix}
\end{equation*}
The objective function \ref{eq:12} can be arranged into a single matrix quadratic form without
the summation as:

\begin{equation}
    \+{x^THx}
\end{equation}
Where:

\begin{equation*}
    \+x = \begin{bmatrix}
        \+x_1 \\
        \+x_2 \\
        \vdots \\
        \+x_{n-1} \\
        \+x_n \\
    \end{bmatrix}
\end{equation*}

\begin{equation*}
    \+H = \begin{bmatrix}
        \+H_1 & \+0 & \hdots & \+0 & \+0 \\
        \+0 & \+H_2 & \hdots & \+0 & \+0 \\
        \vdots & & \ddots & & \vdots \\
        \+0 & \+0 & & \+H_{n-1} & \+0 \\
        \+0 & \+0 & \hdots & \+0 & \+H_n
    \end{bmatrix}
\end{equation*}

\subsection{Minimisation Problem}
The sections above show that finding the maximum smoothness curve comes down to finding
the polynomial coefficients, vector $\+x$, which minimises $\+{x^THx}$, subject to the linear constraints
$\+{Ax=b}$. This problem is well suited to the method of 
Lagrange multipliers for which we first define the vector $\+\lambda$.

\begin{equation}
    \+\lambda = \begin{bmatrix}
        \lambda_1 \\
        \lambda_2 \\
        \vdots \\
        \lambda_{4n-2} \\
        \lambda_{4n-1} \\
    \end{bmatrix}
\end{equation}

\newcommand{\Lagr}{\mathcal{L}}

\begin{equation}
    \Lagr(\+{x, \lambda}) = \+{x^THx + \lambda^T(Ax - b)}
\end{equation}
The minima $\displaystyle\min_{x, \lambda}\Lagr(\+{x, \lambda})$ is found as the solution
where the partial derivatives of $\Lagr(\+{x, \lambda})$ with respect to $\+x$ and $\+\lambda$
are zero.

\begin{equation}
    \frac{\Lagr(\+{x, \lambda})}{\partial{\+x}} = \+{2Hx + A^T\lambda} = 0
\end{equation}

\begin{equation}
    \frac{\Lagr(\+{x, \lambda})}{\partial{\+\lambda}} = \+{Ax - b} = 0
\end{equation}
These can be arranged into a single linear system:

\begin{equation}
    \begin{bmatrix}
        \+{2H} & \+{A^T} \\
        \+{A} & \+{0}
    \end{bmatrix}
    \begin{bmatrix}
        \+{x_{min}} \\
        \+{\lambda_{min}}
    \end{bmatrix} = 
    \begin{bmatrix}
        \+{0} \\
        \+{b}
    \end{bmatrix}
\end{equation}
Hence the vector of spline polynomial coefficients for the maximum smoothness curve, $\+{x_{min}}$,
can be found by solving this system.

\begin{equation}
    \label{eq:final_linear_prob}
    \begin{bmatrix}
        \+{x_{min}} \\
        \+{\lambda_{min}}
    \end{bmatrix} = 
    \begin{bmatrix}
        \+{2H} & \+{A^T} \\
        \+{A} & \+{0}
    \end{bmatrix}^{-1}
    \begin{bmatrix}
        \+{0} \\
        \+{b}
    \end{bmatrix}
\end{equation}

Once \ref{eq:final_linear_prob} is solved, the spline parameters are taken from $\+{x_{min}}$
and the derived forward prices are calculated by evaluating \ref{eq:foward_function}.

% TODO comment on number of solutions and whether this represents a minima or maxima


\bibliographystyle{plainnat}
\bibliography{max_smoothness_spline}

\end{document}